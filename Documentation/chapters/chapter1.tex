\chapter{Introduction}
\label{ch:introduction}


\setlength{\parindent}{0pt} % Remove paragraph indentation

\section{Overview}

Organizations today face an escalating cybersecurity challenge as threat actors become increasingly sophisticated and attack surfaces expand. Security Operations Centers (SOCs) are overwhelmed by the volume of security events generated by diverse monitoring tools, leading to alert fatigue and delayed incident response. Traditional security incident management relies heavily on manual processes, where analysts must correlate information from multiple disparate sources, investigate alerts individually, and coordinate response actions across different systems.

This project presents \textbf{SIREN} (Smart Incident Response \& Event Notifier), an integrated platform designed to streamline security incident detection,\\ analysis, and response. SIREN addresses the critical gap between security event collection and actionable incident response by providing automated \\ correlation, intelligent prioritization, and coordinated notification capabilities within a unified interface.

\vspace{-1.4em}
\section{Problem Definition}
Organizations face critical challenges in security incident management:
\vspace{-1.4em}
\begin{itemize}[itemsep=0.3em, parsep=0pt]
	\item \textbf{Alert Overload:} SOC analysts receive thousands of security alerts daily from multiple sources (SIEM, IDS/IPS, endpoint protection), making it difficult to identify genuine threats among false positives.

	\item \textbf{Manual Correlation:} Security teams must manually correlate events across different tools and time windows to construct complete incident timelines, which is time-consuming and error-prone.

	\item \textbf{Delayed Response:} The time between initial detection and incident response is often measured in hours or days, allowing threats to propagate and cause damage.

	\item \textbf{Fragmented Visibility:} Security data exists in silos across different platforms, preventing comprehensive understanding of the threat landscape and attack patterns.

	\item \textbf{Inefficient Communication:} Incident notifications and response coordination rely on manual processes, leading to communication delays and inconsistent response procedures.
\end{itemize}

These challenges result in increased mean time to detect (MTTD) and mean time to respond (MTTR), leaving organizations vulnerable to security breaches and data loss.

\vspace{-1.4em}
\nopagebreak
\section{Aim}

The aim of this project is to design and develop SIREN, an intelligent security incident response platform that automates event correlation, provides real-time threat intelligence, and enables rapid coordinated response to security incidents through an integrated web-based interface.

\section{Objectives}

\subsection{Primary Objectives}

\begin{enumerate}
	\item \textbf{Automated Event Correlation:} Develop intelligent algorithms to \\ automatically correlate security events from multiple sources and \\identify patterns indicative of genuine security incidents.

	\item \textbf{Real-time Monitoring Dashboard:} Create an interactive web dashboard that provides real-time visibility into security events, active incidents, and system health metrics.

	\item \textbf{Intelligent Alerting System:} Implement a prioritized notification system that alerts appropriate personnel based on incident severity, type, and organizational escalation policies.

	\item \textbf{Integrated Response Workflow:} Design response workflows that enable analysts to investigate, document, and coordinate incident response activities within a single platform.
\end{enumerate}

\subsection{Secondary Objectives}

\begin{enumerate}
	\item \textbf{Python Agent Integration:} Develop a lightweight Python agent for Windows systems that collects security-relevant events and forwards them to the central SIREN platform.

	\item \textbf{Threat Intelligence Enrichment:} Integrate external threat intelligence feeds to provide context and severity scoring for detected events.

	\item \textbf{Historical Analysis:} Implement capabilities for analyzing historical incident data to identify trends and improve detection accuracy.

	\item \textbf{Reporting and Compliance:} Provide automated reporting capabilities for incident documentation and compliance requirements.
\end{enumerate}

\section{Deliverables}

This project will deliver the following components:

\vspace{-1em}
\begin{enumerate}[itemsep=0.3em, parsep=0pt]
	\item \textbf{SIREN Web Application:} Full-stack web application with React TypeScript frontend and Python FastAPI backend providing the core incident management interface.

	\item \textbf{Windows Security Agent:} Python-based agent for Windows systems that monitors security events and forwards them to SIREN.

	\item \textbf{Database Schema:} Comprehensive database design for storing events, incidents, configurations, and user data.

	\item \textbf{API Documentation:} Complete REST API documentation for system integration and extension.

	\item \textbf{User Documentation:} User guides and operational procedures for deploying and operating SIREN.

	\item \textbf{Technical Documentation:} System architecture documentation, deployment guides, and maintenance procedures.

	\item \textbf{Test Results:} Comprehensive testing documentation including functional, performance, and security testing results.
\end{enumerate}

\vspace{-1.5em}
\nopagebreak
\section{Scope}

\vspace{-1.5em}
% \subsection{Within Scope}

\begin{itemize}[itemsep=0.2em, parsep=0pt]
	\item Design and development of web-based incident management platform
	\item Python agent for Windows event collection
	\item Real-time event processing and correlation
	\item Dashboard visualizations and reporting
	\item Alert notification system (email, webhook)
	\item Incident workflow management
	\item User interface for event investigation
	\item Database design and implementation
	      % \item System testing and validation
\end{itemize}

% \subsection{Outside Scope}
%
% \begin{itemize}
% 	\item Integration with commercial SIEM platforms
% 	\item Mobile application development
% 	\item Advanced machine learning threat detection
% 	\item Automated incident remediation
% 	\item Multi-tenant architecture
% 	\item Active Directory integration
% 	\item Network traffic analysis capabilities
% \end{itemize}

\section{Target Customer}

SIREN is designed to serve the following target customers:

\vspace{-1.5em}
\begin{itemize}
	\item \textbf{Small to Medium Enterprises (SMEs):} Organizations with limited \\ security staff who need efficient incident management but lack resources for enterprise SIEM solutions.

	\item \textbf{Managed Security Service Providers (MSSPs):} Security teams managing multiple client environments who require centralized visibility and rapid response capabilities.

	\item \textbf{Educational Institutions:} Universities and colleges seeking affordable security monitoring solutions for campus networks and research environments.

	\item \textbf{Security Operations Centers:} SOC teams requiring supplementary tools for incident correlation and response coordination.
\end{itemize}

\section{Suggested Solution}

SIREN employs a \textbf{four-tier} architecture to deliver comprehensive incident management capabilities:

\vspace{-1.5em}
\subsection{Architecture Components}

\vspace{-1.5em}


\begin{itemize}[itemsep=0.2em, parsep=0pt]
	\item \textbf{Frontend Layer (React TypeScript):} Responsive dashboard with real-time events, incident charts, and threat maps.
	\item \textbf{Backend Layer (Python FastAPI):} Handles event correlation, rule-based detection, incident management, notifications, and queries.
	\item \textbf{Data Layer (PostgreSQL):} Optimized time-series storage for events, incidents, and configurations.
	\item \textbf{Agent Layer (Python Windows Agent):} Lightweight Windows agents monitor logs and forward filtered events to the backend.
\end{itemize}
% \subsection{Key Features}
%
% \begin{itemize}[itemsep=0.2em, parsep=0pt]
% 	\item Event ingestion from multiple sources with normalization
% 	\item Rule-based correlation engine for incident detection
% 	\item Customizable alert thresholds and escalation policies
% 	\item Interactive incident investigation interface
% 	\item Automated notification via email and webhooks
% 	\item Role-based access control for multi-user environments
% 	\item Export capabilities for compliance reporting
% \end{itemize}

\newpage
\section{Gantt Chart}

Figure 1.1 presents the project timeline across the development lifecycle.

\begin{figure}[H]
	\centering
	\includegraphics[width=1\textwidth]{infograhpic.jpg}
	\caption{Project Gantt Chart}
	\label{fig:gantt}
\end{figure}

% \section{Next Chapter Summary}
%
% Chapter 2 examines existing research and commercial solutions related to security incident management, SIEM platforms, and automated response systems. It analyzes three relevant studies, compares their approaches, and identifies gaps that SIREN addresses through its integrated design and focus on SME accessibility.
%
% \section{Gantt Chart}
%
% Figure 1.1 presents the project timeline across the development lifecycle.
%
% \begin{figure}[H]
% 	\centering
% 	\includegraphics[width=1\textwidth]{infograhpic.jpg}
% 	\caption{Project Gantt Chart}
% 	\label{fig:gantt}
% \end{figure}
