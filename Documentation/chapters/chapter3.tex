\chapter{Requirements and System Analysis}
\label{ch:requirements}

\section{Functional Requirements}

Functional requirements define the specific capabilities that SIREN must provide to fulfill its design objectives. These requirements focus on what the system does from the user perspective.

\subsection{FR1: Event Ingestion}
The system shall ingest security events from Python agents deployed on Windows systems, including Windows Event Logs, Sysmon events, and security-relevant application logs. Events must be received via REST API with proper authentication and validation.

\subsection{FR2: Event Normalization}
The system shall normalize events from different sources into a unified schema, extracting common fields (timestamp, severity, source, event type) to enable consistent processing and correlation.

\subsection{FR3: Automated Correlation}
The system shall automatically correlate related security events using configurable rule sets to identify incident patterns such as failed login sequences, privilege escalation attempts, and data exfiltration indicators.

\subsection{FR4: Incident Generation}
When correlation rules are satisfied, the system shall automatically create incident records with severity classification, affected assets, event timeline, and suggested investigation steps.

\subsection{FR5: Real-time Dashboard}
The system shall provide a real-time web dashboard displaying current security status, active incidents, recent events, and system health metrics with auto-refresh capabilities.

\subsection{FR6: Incident Investigation}
The system shall enable analysts to investigate incidents through a dedicated interface showing all correlated events, affected assets, timeline visualization, and investigation notes functionality.

\subsection{FR7: Alerting and Notifications}
The system shall generate alerts for high-severity incidents through multiple channels (email, webhook) based on configurable notification policies and escalation rules.

\subsection{FR8: Incident Lifecycle Management}
The system shall support complete incident lifecycle tracking with status transitions (new, investigating, contained, resolved), assignment to analysts, and resolution documentation.

\subsection{FR9: Reporting}
The system shall generate incident reports in exportable formats (PDF, CSV, JSON) containing incident details, timelines, response actions, and resolution information.

\subsection{FR10: User Authentication}
The system shall implement user authentication with role-based access control distinguishing between administrator, analyst, and viewer permissions.

\section{Technical Requirements}

Technical requirements specify the implementation technologies and architectural constraints.

\subsection{TR1: Web Framework}
Frontend shall be implemented using React 18 with TypeScript for type safety and component reusability.

\subsection{TR2: Backend Framework}
Backend shall utilize Python FastAPI framework for high-performance asynchro nous API services.

\subsection{TR3: Database System}
PostgreSQL shall serve as the primary database for persistent storage of events, incidents, and configuration data.

\subsection{TR4: API Design}
RESTful API architecture shall be employed with JSON request/response formats, supporting standard HTTP methods (GET, POST, PUT, DELETE).

\subsection{TR5: Real-time Communication}
WebSocket or Server-Sent Events shall enable real-time push of incident updates to connected dashboard clients.

\subsection{TR6: Agent Technology}
Windows agents shall be developed in Python 3.8+ with minimal external dependencies to ensure lightweight deployment.

\subsection{TR7: Data Security}
All API communications shall use HTTPS encryption, and sensitive data (passwords, API keys) shall be hashed using industry-standard algorithms.

\section{Business Requirements}

Business requirements address organizational and operational needs beyond technical functionality.

\subsection{BR1: Cost Efficiency}
The solution shall utilize open-source technologies and standard server infrastructure to minimize licensing and deployment costs.

\subsection{BR2: Scalability}
System architecture shall support scaling from small deployments (10-50 endpoints) to medium deployments (500+ endpoints) through horizontal scaling of backend services.

\subsection{BR3: Ease of Deployment}
Deployment shall be achievable by IT staff with standard system administration skills without requiring specialized security engineering expertise.

\subsection{BR4: Maintainability}
System design shall facilitate ongoing maintenance with clear documentation, modular architecture, and standard technology choices.

\subsection{BR5: Compliance Support}
Incident records and audit logs shall support common compliance requirements (SOC 2, ISO 27001) through comprehensive event tracking and reporting.

\section{Non-Functional Requirements}

Non-functional requirements define quality attributes and performance characteristics.

\subsection{NFR1: Performance - Response Time}
The dashboard shall load within 2 seconds under normal conditions, and API endpoints shall respond within 500ms for standard queries.

\subsection{NFR2: Performance - Event Processing}
The correlation engine shall process incoming events with maximum latency of 5 seconds from receipt to incident generation.

\subsection{NFR3: Performance - Concurrent Users}
The system shall support at least 20 concurrent dashboard users without performance degradation.

\subsection{NFR4: Reliability - Availability}
The system shall maintain 99.5\% availability during business hours with graceful degradation when components fail.

\subsection{NFR5: Reliability - Data Integrity}
All security events shall be persisted to database with transaction safety preventing data loss even during system failures.

\subsection{NFR6: Usability}
The interface shall be intuitive enough that trained SOC analysts can perform common tasks (view incidents, investigate events, generate reports) without referring to documentation.

\subsection{NFR7: Maintainability - Code Quality}
Source code shall follow PEP 8 (Python) and ESLint (TypeScript) coding standards with minimum 70\% test coverage.

\subsection{NFR8: Security}
The system shall implement defense-in-depth with input validation, parameterized database queries, session management, and CSRF protection.

\section{Software Requirements}

\subsection{Development Environment}
\begin{itemize}
    \item Python 3.8 or higher
    \item Node.js 16 or higher with npm
    \item PostgreSQL 12 or higher
    \item Git version control
    \item Visual Studio Code or equivalent IDE
\end{itemize}

\subsection{Frontend Dependencies}
\begin{itemize}
    \item React 18.x - UI component framework
    \item TypeScript 4.x - Type-safe JavaScript
    \item TanStack Router - Client-side routing
    \item Recharts - Data visualization
    \item Axios - HTTP client
    \item Tailwind CSS - Styling framework
\end{itemize}

\subsection{Backend Dependencies}
\begin{itemize}
    \item FastAPI 0.95+ - Web framework
    \item SQLAlchemy 2.x - Database ORM
    \item Uvicorn - ASGI server
    \item Pydantic - Data validation
    \item python-jose - JWT authentication
    \item Alembic - Database migrations
\end{itemize}

\subsection{Agent Dependencies}
\begin{itemize}
    \item Python 3.8+ (Windows compatible)
    \item pywin32 - Windows API access
    \item requests - HTTP client
    \item schedule - Task scheduling
\end{itemize}

\section{Hardware Requirements}

\subsection{Development System}
\begin{itemize}
    \item CPU: Dual-core 2.0 GHz minimum
    \item RAM: 8 GB minimum
    \item Storage: 50 GB available space
    \item Network: Standard ethernet connection
\end{itemize}

\subsection{Production Server}
\begin{itemize}
    \item CPU: Quad-core 2.5 GHz recommended
    \item RAM: 16 GB minimum, 32 GB recommended
    \item Storage: 200 GB SSD for database and logs
    \item Network: 1 Gbps network interface
\end{itemize}

\subsection{Windows Agent (Per Endpoint)}
\begin{itemize}
    \item CPU: Minimal impact (<5\% CPU usage)
    \item RAM: 100 MB footprint
    \item Storage: 50 MB for agent and logs
    \item Network: Standard network connectivity
\end{itemize}

\section{UML Diagrams}

\subsection{Data Flow Diagram}

Figure \ref{fig:dfd} illustrates how security event data flows through the SIREN system from collection to analyst presentation.

\begin{figure}[H]
\centering
\begin{verbatim}
  Windows                                    SIREN
 Endpoints                              Backend Server              Web Browser
+---------+                           +-----------------+         +-----------+
|         |                           |                 |         |           |
| Python  |---(1) Events via HTTPS--->| FastAPI         |         | React     |
| Agent   |                           | Event Receiver  |         | Dashboard |
|         |                           |                 |         |           |
| Monitors:|                          +--------+--------+         +-----+-----+
| - Event  |                                   |                        ^
|   Logs   |                                   v                        |
| - Sysmon |                          +--------+--------+               |
| - Apps   |                          | Correlation     |               |
+---------+                           | Engine          |               |
                                      |                 |               |
                                      | - Rule Matching |               |
                                      | - Pattern Detect|               |
                                      +--------+--------+               |
                                               |                        |
                                               v                        |
                                      +--------+--------+               |
                                      | Incident        |               |
                                      | Generator       |               |
                                      +--------+--------+               |
                                               |                        |
                                               v                        |
                                      +--------+--------+               |
                                      | PostgreSQL DB   |               |
                                      | - Events        |               |
                                      | - Incidents     |               |
                                      | - Config        |               |
                                      +--------+--------+               |
                                               |                        |
                                               v                        |
                                      +--------+--------+               |
                                      | Alert/Notify    |               |
                                      | Service         |               |
                                      +--------+--------+               |
                                               |                        |
        (2) Email/Webhook<--------------------+                        |
                                                                        |
        (3) Dashboard Query API ----------------------------------------+
\end{verbatim}
\caption{Data Flow Diagram showing event processing pipeline}
\label{fig:dfd}
\end{figure}

\textbf{Data Flow Explanation:}

\begin{enumerate}
    \item \textbf{Event Collection:} Python agents on Windows endpoints continuously monitor security event sources and forward events to SIREN backend via HTTPS POST requests.
    
    \item \textbf{Event Processing:} The FastAPI Event Receiver validates and normalizes incoming events, storing them in PostgreSQL.
    
    \item \textbf{Correlation:} The Correlation Engine continuously evaluates events against configured rules, identifying patterns indicative of security incidents.
    
    \item \textbf{Incident Generation:} When correlation rules match, the Incident Generator creates incident records with severity, classification, and affected assets.
    
    \item \textbf{Alert Notification:} The Alert Service sends notifications through configured channels (email, webhook) for high-severity incidents.
    
    \item \textbf{Dashboard Presentation:} React frontend queries the API to retrieve and display incidents, events, and system metrics in real-time.
\end{enumerate}

\subsection{Use Case Diagram}

Figure \ref{fig:usecase} depicts primary user interactions with the SIREN system.

\begin{figure}[H]
\centering
\begin{verbatim}
                                    SIREN System
                      +---------------------------------------+
                      |                                       |
     SOC Analyst      |   [View Dashboard]                    |
         o            |         |                             |
        /|\           |         v                             |
        / \           |   [Investigate Incident]              |
         |----------->|         |                             |
         |            |         +--->[View Event Timeline]    |
         |            |         |                             |
         |            |         +--->[Document Findings]      |
         |            |         |                             |
         |            |         +--->[Update Status]          |
         |            |                                       |
         |            |   [Generate Report]                   |
         |----------->|                                       |
         |            |                                       |
         |            |   [Configure Correlation Rules]       |
         +----------->|         |                             |
                      |         v                             |
   Administrator      |   [Manage Users]                      |
         o            |                                       |
        /|\           |   [Configure Alerts]                  |
        / \           |         |                             |
         +----------->|         +--->[Email Settings]         |
         |            |         |                             |
         |            |         +--->[Webhook Settings]       |
                      |                                       |
  Windows Agent       |   [Send Events]                       |
     (System)         |         |                             |
         |            |         v                             |
         +----------->|   [Receive Events]                    |
                      |         |                             |
                      |         v                             |
                      |   [Correlate & Alert]                 |
                      |                                       |
                      +---------------------------------------+
\end{verbatim}
\caption{Use Case Diagram showing actor interactions}
\label{fig:usecase}
\end{figure}

\textbf{Use Case Explanation:}

\textbf{SOC Analyst Use Cases:}
\begin{itemize}
    \item \textbf{View Dashboard:} Displays real-time security status, active incidents, and recent events.
    \item \textbf{Investigate Incident:} Drill into specific incidents to view correlated events, timeline, and affected assets.
    \item \textbf{Generate Report:} Export incident documentation for compliance or communication purposes.
\end{itemize}

\textbf{Administrator Use Cases:}
\begin{itemize}
    \item \textbf{Configure Correlation Rules:} Define patterns that trigger incident generation.
    \item \textbf{Manage Users:} Create, modify, and disable user accounts with appropriate permissions.
    \item \textbf{Configure Alerts:} Set up email and webhook notification channels.
\end{itemize}

\textbf{System Actor (Windows Agent):}
\begin{itemize}
    \item \textbf{Send Events:} Automated process forwarding security events to SIREN backend.
\end{itemize}

\subsection{Activity Diagram}

Figure \ref{fig:activity} illustrates the workflow for incident detection and response.

\begin{figure}[H]
\centering
\begin{verbatim}
                        [Start: Event Occurs on Endpoint]
                                      |
                                      v
                        +------------------------+
                        | Agent Captures Event   |
                        +------------------------+
                                      |
                                      v
                        +------------------------+
                        | Filter Event Locally   |
                        +------------------------+
                                      |
                          +-----------+-----------+
                          |                       |
                      [Relevant?]             [Not Relevant]
                          |                       |
                          v                       v
                    +-------------+          [Discard]
                    | Send to     |
                    | SIREN API   |
                    +-------------+
                          |
                          v
                    +-------------+
                    | Normalize & |
                    | Store Event |
                    +-------------+
                          |
                          v
                    +-------------+
                    | Run Rules   |
                    | Engine      |
                    +-------------+
                          |
              +-----------+-----------+
              |                       |
        [Rule Match]            [No Match]
              |                       |
              v                       v
        +-------------+         [Continue
        | Generate    |          Monitoring]
        | Incident    |
        +-------------+
              |
              v
        +-------------+
        | Determine   |
        | Severity    |
        +-------------+
              |
        +-----+-----+
        |           |
    [High/     [Medium/
     Critical]  Low]
        |           |
        v           v
    +-------+   +-------+
    | Send  |   | Log   |
    | Alert |   | Only  |
    +-------+   +-------+
        |           |
        +-----------+
              |
              v
        +-------------+
        | Analyst     |
        | Investigates|
        +-------------+
              |
              v
        +-------------+
        | Document &  |
        | Resolve     |
        +-------------+
              |
              v
        [End: Incident Closed]
\end{verbatim}
\caption{Activity Diagram for incident detection and response workflow}
\label{fig:activity}
\end{figure}

\textbf{Activity Flow Explanation:}

\begin{enumerate}
    \item \textbf{Event Capture:} Windows agent detects security event (login attempt, file access, process execution).
    
    \item \textbf{Local Filtering:} Agent applies basic filters to reduce network traffic by discarding obviously benign events.
    
    \item \textbf{Event Transmission:} Relevant events are sent to SIREN backend via authenticated API call.
    
    \item \textbf{Normalization:} Events are normalized into standard schema and stored in database.
    
    \item \textbf{Rule Evaluation:} Correlation engine evaluates events against configured detection rules.
    
    \item \textbf{Incident Generation:} When rules match (e.g., 5 failed logins in 2 minutes), an incident is created.
    
    \item \textbf{Severity Classification:} Incident severity is determined based on rule configuration and event characteristics.
    
    \item \textbf{Alert Distribution:} High-severity incidents trigger immediate alerts to SOC analysts.
    
    \item \textbf{Investigation:} Analysts review incidents, examine correlated events, and determine appropriate response.
    
    \item \textbf{Resolution:} Incident is documented and marked as resolved after containment and remediation.
\end{enumerate}

\section{Tools Used}

\subsection{Development Tools}
\begin{itemize}
    \item \textbf{Visual Studio Code:} Primary IDE for both frontend and backend development
    \item \textbf{Git:} Version control and collaborative development
    \item \textbf{Postman:} API testing and documentation
    \item \textbf{pgAdmin:} PostgreSQL database administration
    \item \textbf{Chrome DevTools:} Frontend debugging and performance analysis
\end{itemize}

\subsection{Testing Tools}
\begin{itemize}
    \item \textbf{Jest:} Frontend component and unit testing
    \item \textbf{pytest:} Backend unit and integration testing
    \item \textbf{React Testing Library:} UI component testing
    \item \textbf{Locust:} Performance and load testing
\end{itemize}

\subsection{Deployment Tools}
\begin{itemize}
    \item \textbf{Docker:} Containerization for consistent deployment
    \item \textbf{Docker Compose:} Multi-container orchestration
    \item \textbf{Nginx:} Reverse proxy and static file serving
\end{itemize}

\section{Code of Ethics}

The development and deployment of security incident management systems raises important ethical considerations that have been carefully addressed throughout this project.

\subsection{Privacy and Data Protection}

Security event data often contains sensitive information about user activities, system configurations, and organizational infrastructure. This project adheres to principles of data minimization and purpose limitation:

\begin{itemize}
    \item Only security-relevant events are collected, avoiding unnecessary personal data capture
    \item Data retention policies limit storage duration to operational necessity
    \item Access controls ensure only authorized personnel can view sensitive event data
    \item Incident reports can be anonymized when shared outside immediate security teams
\end{itemize}

\subsection{Transparency and Accountability}

Users and administrators must understand how SIREN operates:

\begin{itemize}
    \item Correlation rules are explicit and documented, not opaque algorithmic decisions
    \item Audit logs track all system activities including user actions and automated processes
    \item Incident generation rationale is clearly presented, showing which events triggered which rules
    \item System limitations are documented honestly without overstating detection capabilities
\end{itemize}

\subsection{Security by Design}

The system itself must exemplify security best practices:

\begin{itemize}
    \item Secure coding practices prevent common vulnerabilities (SQL injection, XSS, CSRF)
    \item Authentication and authorization protect against unauthorized access
    \item Encryption protects data in transit and sensitive data at rest
    \item Regular security testing identifies and addresses potential weaknesses
\end{itemize}

\subsection{Responsible Disclosure}

Any vulnerabilities discovered during development have been:

\begin{itemize}
    \item Documented and addressed before deployment
    \item Reported to relevant parties when affecting third-party dependencies
    \item Never exploited or disclosed publicly without appropriate timeline for remediation
\end{itemize}

\subsection{Avoid Harm}

Security tools can be misused for surveillance or oppression. This project includes safeguards:

\begin{itemize}
    \item Documentation emphasizes lawful and ethical use within appropriate organizational context
    \item No capabilities designed specifically for mass surveillance or privacy invasion
    \item Open-source nature enables audit and verification of functionality
    \item User authentication and audit logging create accountability for system use
\end{itemize}

\subsection{Professional Responsibility}

Development adhered to professional engineering principles:

\begin{itemize}
    \item Honest representation of system capabilities and limitations
    \item Acknowledgment of existing work and proper attribution
    \item Comprehensive testing before deployment
    \item Clear documentation for users and maintainers
    \item Commitment to addressing discovered issues responsibly
\end{itemize}

The ethical framework guiding this project recognizes that security tools exist within broader social and legal contexts. SIREN is designed to enhance legitimate organizational security while respecting individual privacy, maintaining transparency, and enabling accountability. These ethical considerations are not afterthoughts but foundational principles integrated throughout the design, implementation, and deployment processes.
