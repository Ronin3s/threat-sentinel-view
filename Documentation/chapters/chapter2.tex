\chapter{Review of Tools and Related Work}
\label{ch:literature}

\section{Overview}

This chapter examines existing research, tools, and commercial solutions relevant to security incident management, automated event correlation, and Security Operations Center (SOC) platforms. The review analyzes three significant studies that address challenges similar to those SIREN aims to solve. Each study is evaluated based on its methodology, approach, and contributions, with comparative analysis identifying similarities and differences. This critical examination establishes the context for SIREN's design decisions and highlights gaps in existing solutions that this project addresses.

\section{Related Work}

\subsection{Study 1: Automated Incident Response Using SOAR }

\textbf{Reference:} Zimmerman et al. (2019), "Security Orchestration, Automation and Response: A Systematic Review of SOAR Capabilities and Implementations"

\subsubsection{Methodology}

\vspace{-1.4em}

Zimmerman et al. surveyed 45 enterprises to evaluate SOAR platform effectiveness, combining quantitative incident metrics (MTTD, MTTR) with qualitative analyst interviews. The study measured the impact of automation on alert volumes, false positive rates, and response times across various incident categories, comparing manual versus automated workflows in controlled test environments.

\subsubsection{Similarities to SIREN}

\vspace{-1.4em}
SIREN shares core objectives with SOAR platforms:
\begin{itemize}[itemsep=0.1em, parsep=0pt]
	\item \textbf{Alert Consolidation:} Integrates alerts for pattern detection.
	\item \textbf{Workflow Automation:} Reduces analyst workload.
	\item \textbf{Centralized Management:} Provides a unified monitoring interface.
\end{itemize}

\subsubsection{Differences from SIREN}

Key distinctions include:
\begin{itemize}[itemsep=0.2em, parsep=0pt]
	\item \textbf{Target \& Complexity:} SOAR targets enterprises with complex, expensive solutions requiring specialized training. SIREN targets SMEs with a simple, pre-configured, and intuitive approach.
	\item \textbf{Cost \& Access:} SOAR platforms have high licensing costs (\$50k+), whereas SIREN is a \textbf{free, open-source solution.}
	\item \textbf{Scope:} SOAR integrates with vast enterprise ecosystems; \\SIREN focuses on essential Windows event sources for straightforward,\\ lightweight deployment.
\end{itemize}

\subsection{Study 2: Machine Learning for Security Incident Detection}

\textbf{Reference:} Chen et al. (2020), "Deep Learning Approaches for Anomaly Detection in Security Information and Event Management Systems"

\subsubsection{Methodology}

\vspace{-1.5em}
Chen et al. evaluated deep learning models (RNN, LSTM, CNN) for incident detection using 3 months of data (50M daily events) from a university network. They labeled 100k events for training and measured accuracy, false positives, and latency using k-fold cross-validation against both historical and live traffic.

\subsubsection{Similarities to SIREN}

\vspace{-1.5em}
Common goals include:

\vspace{-1.5em}
\begin{itemize}[itemsep=0.1em, parsep=0pt]
	\item \textbf{Pattern Recognition:} Identifying complex threat patterns in event streams.
	\item \textbf{Automation:} Reducing manual analysis through automated detection.
	\item \textbf{Scale \& Accuracy:} Processing high volumes efficiently while minimizing false positives.
\end{itemize}

\subsubsection{Differences from SIREN}

\vspace{-1.5em}
Key distinctions include:

\vspace{-1.5em}
\begin{itemize}[itemsep=0.1em, parsep=0pt]
	\item \textbf{Detection Approach:} ML relies on opaque "black box" models requiring extensive labeled training data. SIREN uses transparent, rule-based correlation without training requirements.
	\item \textbf{Resources:} ML demands heavy computation; SIREN is lightweight and suitable for SMEs.
	\item \textbf{Adaptability:} ML requires retraining for new threats; SIREN allows instant rule updates.
\end{itemize}

\subsection{Study 3: Open-Source SIEM Platform Architecture}

\textbf{Reference:} Patel and Kumar (2021), "Design and Implementation of Scalable Open-Source Security Information and Event Management Platform"

\subsubsection{Methodology}

\vspace{-1.5em}
Patel and Kumar developed an open-source SIEM using the ELK stack, refining it through action research with three pilot organizations. They evaluated performance (ingestion rates up to 100k EPS), storage optimization, and user experience through analyst interviews.

\subsubsection{Similarities to SIREN}

\vspace{-1.5em}
Shared architectural philosophies include:

\vspace{-1.5em}
\begin{itemize}[itemsep=0.1em, parsep=0pt]
	\item \textbf{Open-Source & Web-Based:} Both offer accessible, cost-free solutions with modern web interfaces.
	\item \textbf{Centralization:} Both consolidate events from distributed sources for unified analysis.
	\item \textbf{Flexibility:} Both emphasize customization and time-series data optimization.
\end{itemize}

\subsubsection{Differences from SIREN}

\vspace{-1.5em}
Key distinctions include:

\vspace{-1.5em}
\begin{itemize}[itemsep=0.1em, parsep=0pt]
	\item \textbf{Scope:} SIEM is broad (log search); SIREN is focused (incident correlation & response).
	\item \textbf{Response:} SIREN integrates response workflows; SIEM relies on external tools.
	\item \textbf{Agent:} SIREN uses a specialized lightweight agent; SIEM uses general-purpose forwarders.
\end{itemize}

\section{Comparison Table}

\begin{table}[H]
	\centering
	\caption{Comparison of Related Research and SIREN}
	\label{tab:comparison}
	\small
	\begin{tabularx}{\textwidth}{|l|X|X|X|X|}
		\hline
		\textbf{Dimension}    & \textbf{SOAR Platforms} & \textbf{ML Detection} & \textbf{Open-Source SIEM} & \textbf{SIREN}         \\
		\hline
		Target Users          & Large Enterprise        & Research/Large Org    & Flexible                  & SME/MSSP               \\
		\hline
		Detection Method      & Rule-based + Playbooks  & Deep Learning         & Query-based               & Rule-based Correlation \\
		\hline
		Complexity            & High                    & Very High             & Moderate                  & Low                    \\
		\hline
		Cost                  & \$50K-\$500K+           & Development Cost      & Free (Infrastructure)     & Free                   \\
		\hline
		Training Required     & Extensive               & Model Training Needed & Moderate                  & Minimal                \\
		\hline
		Deployment Time       & Weeks-Months            & Months                & Days-Weeks                & Days                   \\
		\hline
		Explainability        & High                    & Low                   & High                      & High                   \\
		\hline
		Response Integration  & Extensive               & None                  & Limited                   & Integrated             \\
		\hline
		Resource Requirements & Heavy                   & Very Heavy            & Heavy                     & Moderate               \\
		\hline
		Customization         & Complex                 & Model Retraining      & Query DSL                 & Configuration Rules    \\
		\hline
	\end{tabularx}
\end{table}

\begin{figure}[H]
	\centering
	\includegraphics[width=1.1\textwidth, height=25cm, keepaspectratio]{comper.png}
	\caption{Comparison Visualization}
	\label{fig:comparison_visual}
\end{figure}

% \begin{figure}[H]
% 	\centering
% 	\includegraphics[width=1.2\textwidth]{comper.png}
% 	\caption{Comparison Visualization}
% 	\label{fig:comparison_visual}
% \end{figure}


\begin{figure}[H]
	\centering
	\includegraphics[width=0.9\textwidth]{sec-gap-analsiys.png}
	\caption{Gap Analysis: How SIREN Addresses Under-Served Market Segment}
	\label{fig:gap_analysis}
\end{figure}

Figure \ref{fig:gap_analysis} illustrates that existing solutions—enterprise SOAR platforms, ML-based detection systems, and open-source SIEM platforms—each have limitations preventing SME adoption. SIREN fills this gap by balancing capability, simplicity, and cost-effectiveness.

% \section{Summary}
%
% This chapter reviewed existing approaches to security incident management, including enterprise SOAR platforms, machine learning-based detection, and open-source SIEM architectures. While these solutions demonstrate the value of automation and integration, they present barriers to adoption by resource-constrained organizations through high costs, complexity, or limited functionality.
%
% SIREN addresses these gaps by combining workflow automation from SOAR platforms, explainable rule-based correlation instead of resource-intensive machine learning, and the accessibility principles of open-source tools—while maintaining tight focus on incident management and response. This positions SIREN uniquely for SME organizations requiring effective security operations without enterprise-level complexity or costs.
%
